\documentclass[11pt]{article}

% GILDAS specific definitions
\include{gildas-def}

\title {Flux measurements with the \\
  IRAM Plateau de Bure Interferometer}%
\author {A.Dutrey$^1$, S.Guilloteau$^1$}%
\date{Document probably older than you think}%
\makeindex

%%%%%%%%%%%%%%%%%%%%%%%%%%%%%%%%%%%%%%%%%%%%%%%%%%%%%%%%%%%%%%%%%%%%%%%%%%%

\begin{document}

\maketitle{}

\begin{center}
  Version 1.2
\end{center}

\begin{center}
  (1) Institut de Radio Astronomie Millim\'etrique\\
  300 Rue de la Piscine\\
  F-38406 Saint Martin d'H\`eres
\end{center}

This document sums up the methods for flux measurements which are available
at the IRAM Plateau de Bure Interferometer.  These methods are still
experimental and under development and improvements.  All comments from
enthusiastic or angry astronomers and/or operators are welcome.

\begin{rawhtml}
  Note: this is the on-line version of the "Flux Measurements" A <A
  HREF="../../pdf/lclass.pdf"> PDF version</A> is also available.
\end{rawhtml}

Related information is available in:
\begin{latexonly}
\begin{itemize}
\item{CLASS: Continuum and Line Analysis Simple Software}
\item{CLIC: Continuum and Line Interferometric Calibration}
\item{ASTRO: Astronomical Software To pRepare Observations}
\item{GILDAS: Grenoble Image and Line Data Analysis Software}
\end{itemize}
\end{latexonly}

\begin{rawhtml}
  <UL>
  <LI> IRAM Plateau de Bure Interferometer: <A HREF="../clic-html/clic.html"> Continuum and Line Interferometer Calibration</A>
  <LI> <A HREF="../class-html/class.html"> Continuum and Line Analysis Single-dish Software</A>
  <LI> <A HREF="../astro-html/astro.html"> ASTRO:</A>  A Software to pRepare Observations
  <LI> <A HREF="../greg-html/greg.html"> GreG:</A> Grenoble Graphics
  <LI> <A HREF="../gildas-intro-html/gildas-intro.html"> GILDAS:</A> Grenoble Image and Line Data Analysis Software
\end{rawhtml}

Revision 1.0: Basic documentation\\
Revision 1.1: Introduction of program FLUX.\\
Revision 1.2: Determination of flux scale for a project\\
Revision 2.0: Coordination with 30-m (in preparation).

\newpage
\tableofcontents

\newpage

\section{Introduction}

\subsection{The Need for Flux Calibration}

Because of the variable amount of decorrelation introduced by phase noise,
an interferometer cannot easily measure absolute fluxes. All measurements
are relative to some source of known flux. In practice, planets are used
because they are among the few astronomical objects sufficiently strong at
millimeter wavelength for which flux density predictions are possible and
with

Planets are too big to be measured directly in interferometry, except for
the smallest at short baselines. Thus, single-dish measurements are used to
bootstrap the flux of strong quasars (point sources) from the planets.
Weaker sources are then referred to stronger ones in interferometry because
single-dish measurements are not sensitive enough.

Since the quasars are highly variable, we need to observe at least the flux
from the strongest of them (single-dish measurements) each month.  Because
of this variability, an error in the flux scale during one configuration
does not result in a simple scale factor in the final image, but introduces
severe artefacts.

Accordingly, flux measurements have to be considered as a HIGH priority
project on the interferometer, since they may ultimately limit the image
quality. Special observing dates are selected from the best possibilities
offered by the planets visibilities.  In order to interpolate the flux for
all the frequencies available at the Plateau de Bure Interferometer, we
need one frequency at each edge of the frequency band to derive the
spectral index of each source.


\subsection{Calibration Procedure}

Because of the physics of quasars, the spectral index may be variable with
time as the source intensity. Almost simultaneous measurements at 2
frequencies are thus needed to estimate it accurately. In practice, it is
impossible to follow all the $\simeq$ 100 quasars used as phase reference
at the IRAM interferometer. Even following the $\simeq$ 20 ones typically
used during a 4 month observing period is a time consuming task.

Accordingly, only a handful of strong quasars are monitored frequently, at
2 frequencies. These are the strongest quasars, typically used for bandpass
calibration by all projects. A complementary flux check is done using the
continuum source W3OH. As it is partially resolved by the interferometer, a
model of the emission is however necessary.  This additional procedure will
be available soon.


To make the monitoring of strong quasars easy, for the observations and for
the analysis, but also to minimize the calibration errors, a standard
observing procedure has been devised.  The two frequencies selected
correspond to the SiOv1 line at 86 GHz and 13CO(2-1) line at 210 GHz. As
the calibration accuracy is a crucial point, fluxes must be measured only
by very good weather.

Of course, short ``holes'' (1 or 2 hours) in the observations by good
weather can also be used to observe fluxes using the current frequency
tuning but have to be considered as ``COMPLEMENTARY'' data. One need to
obtain a few precise measurements at fixed frequencies rather than lots of
data taken in incertain conditions.

 
\section{Observing procedure}

The procedure {\tt FLUX} (corresponding to the project {\tt FLUX}) is
similar to {\tt BASE} or {\tt POINT} and runs under the \obs\ program.
This procedure makes flux measurements in single-dish, multi-dish or
interferometric mode. For commodity and a better calibration accuracy, it
uses only the two following frequencies: SiOv1 at 86 GHz and 13CO(2-1) at
210 GHz. A third option allows using the current frequency tuning for only
SMALL ``holes'' between two projects.

To start it, just type {\tt @ PR:FLUX} then answer the questions about
frequency choice, observing mode and so on...

\subsection{ Single and Multi-dish mode}

Observations in multi and single-dish modes are made using pointings
cross-scans in beam-switching mode after a careful CALIBRATION at the
source elevation.  Calibrations on planets are made off source to avoid
contamination by source emission (although even Saturn contributes to only
6 K to the total noise).  Skydips at the beginning and/or end of
observations can provide us a better estimate of the calibration accuracy.
Several pointing scans may be done at various elevations.  Before starting
the observations, tests of the beam-switch and the value (or measure) of
the Gains USB/LSB are automatically asked.  The gain value is used with USB
and LSB frequencies to calculate a weighted flux of the planets, including
the gain USB/LSB.  If it is impossible to measure the gain correctly (1 or
2 antennas only), be careful about the value you enter (ask the operator on
duty).

The planets are selected in priority at the beginning of the observations,
a flag counts the planets measurements and signals when no planet is
available.  Afterwards, only the brightest quasars are observed :

\begin{flushleft}
  {Table 1: Strong sources observed in single-dish mode (Apr-1992).}
  \vskip 0.1 cm
\begin{tabular}{|l|l|}
\hline
\hline
3C84 & \\
3C273 & \\
3C279 & (1) \\
3C345 & \\
3C454.3 & \\
NRAO530 & (2) \\
0420-014 & \\
0851+202 & (2) \\
\hline
W3OH & \\
\hline
\hline
\end{tabular}
\vskip 0.1 cm
(1) Not used, too close to 3C273. \\
(2) Sometimes used, but may be too weak. \\
\end{flushleft}

\subsection{ Interferometric mode}

Observations in the interferometric mode can be done using either
cross-correlations or pointing scans. Using pointing scans, a correction
from pointing errors is possible (see section ``Reduction method'').  The
procedure asks for the measurements of the gain USB/LSB value (same remark
as above).

In addition to the brightest quasars mentionned above, several phase
calibrators (secondary flux calibrators) are also selected by this
procedure.  The list is variable, but typically includes a few weaker
sources frequently used on observing projects. For example in April 1992:

\begin{flushleft}
  {Table 2: Secondary calibrators observed in interferometric mode} \vskip
  0.1 cm
\begin{tabular}{|l|}
\hline
\hline
0851+202\\
NRAO530\\
2200+420\\
3C120\\
\\
\\
\\
\hline
\hline
\end{tabular}
\end{flushleft}

\section{Reduction method}

\subsection{Single-dish}

Single-dish data are analysed using \class\ by Gaussian fits with the
method {\tt CONTINUUM Beam\_size} which takes into account the dual-beam
pattern in elevation. Beam width, beam separation and beam ratios are fixed
for better sensitivity. One planets, these parameters are freed using the
{\tt GAUSS * * *} command, since the planets are marginally resolved. The
beam-size is 56 $''$ at 86 GHz.

A typical sequence of single-dish flux reduction is the following:
\begin{verbatim}

FIL IN 'date'.BUR
FIL OUT 'flux-date'.BUR      !
@ PR:INIT_FLUX Teles Line    ! Initialize once
R Scan
...
R Scan      ! Repeat for all scans in the file
!
@ PR:INIT_FLUX Teles Line    ! Repeat for other frequencies/Telescopes
...
R Scan

\end{verbatim}

The setup procedure {\tt PR:INIT\_FLUX} contains
\begin{verbatim}

! @ PR:INIT_FLUX Telescope Line
! Initialize a flux data reduction
LAS\SET TEL &1           ! Select Telescope 
LAS\SET LINE &2          ! Select Observed Line
LAS\SET TYPE CONTINUUM
LAS\SET ALIGN POSITION
LAS\SET ANGLE SECOND
DEVICE                   ! Enter graphics device
SIC\DEFINE REAL F
SAY "Enter LO1 frequency in GHz (0 if not known or variable)"
LET F
IF (F.GT.80.AND.F.LT.118) THEN
   Say "The beam-size is:" '86*56|F'  
   ANALYSE\METHOD CONTINUUM 86*56|F     ! Fix the beam width
ELSE
   ANALYSE\METHOD CONTINUUM             ! Let it free if frequency not known
ENDIF
SYMBOL R "SIC\@ PR:FLUX"
\end{verbatim}

{\tt R} is a symbol calling the reduction procedure {\tt PR:FLUX.CLASS}
\begin{verbatim}
!
! CLASS : Reduce a Pointing cross scan to get flux measurements
!
! Calling sequence : @ PR:INIT_FLUX
! Then R Scan_Number (R = "@ PR:FLUX")
!
DEFINE INTEGER ISCAN
DEFINE LOGICAL PLANET
LET ISCAN &1
LAS\FIND/SCAN ISCAN
LET PLANET (SOURCE.EQ."JUPITER".OR.SOURCE.EQ."VENUS".OR.SOURCE.EQ."MARS".OR.-
SOURCE.EQ."SATURN") 
!
LAS\GET FIRST
LAS\GET NEXT
LAS\ACCUMULATE
SET MOD X -110 110
LAS\PLOT
SIC\IF PLANET THEN
   ANALYSE\GAUSS * * *
ELSE
   ANALYSE\GAUSS
ENDIF
ANALYSE\FIT
PAUSE   "WRITE the result if it seems OK"
LAS\GET NEXT
LAS\GET NEXT
LAS\ACCUMULATE
SET MOD X -110 230
LAS\PLOT
SIC\IF PLANET THEN
   ANALYSE\GAUSS * * *
ELSE
   ANALYSE\GAUSS
ENDIF
ANALYSE\FIT
PAUSE "WRITE the result if it seems OK"
SET MOD X -110 110
\end{verbatim}

When all the reduction is finished, print the fit results from the output
file
\begin{verbatim}
FIL IN 'flux-date'.BUR
FIND
PRINT FLUX/OUT 'bur*-frequency-date'.FIT
\end{verbatim}

The Gaussian fit results are written in a special file (produced by the new
command {\class\ \tt PRINT FLUX /OUTPUT Filename}) which contains all the
fit parameters, the antenna number, the signal and image frequencies, the
gain USB/LSB and the elevation of the source.

This file is then directly analysed using \astro\ with the new language
FLUX which corresponds to the commands described in section 4.

\subsection{Interferometry}

For cross-correlations, the data reduction is made using \clic . The
following sequence is used
\begin{verbatim}
CLIC\FILE BO 'name'.IPB
CLIC\SET PROJECT FLUX
CLIC\FIND
CLIC\DEVICE
CLIC\SET BAN USB LSB
CLIC\SET SUBBANDS C01 TO C10
CLIC\SET X I_FREQUENCY
CLIC\SET Y AMPLITUDE PHASE
! Calibrate RF passband to use DSB mode
CLIC\FIND/PROC COR/SOUR 'Strong_Quasar'
CLIC\PLOT
CLIC\SOLVE RF/PL
CLIC\FIND/PROC CORR/SOUR *
CLIC\STORE RF
CLIC\SET RF ON CHAN
! Use DSB unless gain ratio is very different from 1
CLIC\SET BAN DSB
CLIC\SET FLUX 'Quasar' Flux    ! Specify flux of reference quasar
CLIC\SET FLUX 'Quasar' Flux    ! More than one allowed
CLIC\SOLVE FLUX
CLIC\PRINT FLUX
\end{verbatim} 

For pointing scans, the option /FLUX file in the command:
\begin{verbatim}
CLIC\SOLVE POINTING [/PLOT] /FLUX file
\end{verbatim}
processes an output file containing the same information and using the same
format as the single-dish flux file created by \class .  This file may then
read directly by the FLUX program.

\section{How to Determine the Absolute Flux Scale on a Project}

Contrarily to a common idea, determining the absolute flux scale in an
interferometric project is the hardest and finest task of the calibration.
Furthermore the {\tt CLIC} command {\tt SOLVE FLUX} is very dangerous,
particularly when used on bad calibrated data.

This command permits the determination of the absolute flux scale which is
fixed by the {\tt CLIC} command {\tt STORE}.

During this step, all parameters which can vary in time have a critical
influence on the quality of the final results: POINTING, FOCUS, weather
(amplitude and phase noise) and can introduce somes biases. POINTING errors
are the most important but can be easily cancelled by pointing before doing
the cross-correlations on the RF AND the PHASE CALIBRATOR. The evolution in
time of the phase noise is most difficult to cancel.  For example, a phase
noise fluctuation from 15$^o$ to 30$^o$ between the RF bandpass calibrator
and the phase calibrator introduces on the efficiencies estimates a
variation of about $\sim$ 5.5 \%.

In order to avoid them, the best solution is to apply the SOLVE FLUX
command only on a short time interval where the weather conditions: H$_2$O
vapor contents and phase noise are similar (compare the CALIBRATIONS).

In standard interferometric projects, we find the following data:
 
\begin{verbatim}
        - POINTING on RF Bandpass Calibrator
        - CROSS-CORELATION on RF Bandpass Calibrator (typically 1/4h)
        - POINTINGS on Phase Calibrator (each 2h)
        - CROSS-CORELATION (by 4min) on Phase Calibrator 
\end{verbatim}

We us them to determine the absolute flux scale.  Depending of the flux of
the amplitude (phase) calibrator, there are several possibilities.

\subsection{The Amplitude Calibrator is used for Pointings} 
 
In this case, the flux of the amplitude calibrator will be determined from
pointings using the RF bandpass calibrator.  The method is the same as
described in section 3. After doing standard RF and phase calibration, one
need to reduce the pointing data and write the results in a file. The
sequence is the following:

\begin{verbatim}
        CLIC\SET RF ON FREQ
        CLIC\SET PHASE RELATIVE
        CLIC\SET BAND DSB
        CLIC\SET SUB C01 TO C06
        CLIC\SET SOURCE Phase_Calib RF_CAlib
        CLIC\SET PROC POINT
        CLIC\FIND
        CLIC\SOLVE POINTING/[PLOT]/OUTPUT flux.fit FLUX NEW
        CLIC\EXIT
\end{verbatim}

The command
\begin{verbatim}
CLIC\SOLVE POINTING [/PLOT] /OUTPUT file FLUX NEW[OLD]
\end{verbatim}
processes an output file which can be directly read by the language FLUX in
{\tt ASTRO}. The section 5 explains how to use this file to derive the flux
density of the amplitude calibrator using the flux of the Primary
calibrator (which is given in the flux reports).

When the flux density of the amplitude calibrator is known, update the {\tt
  CLIC} calibration procedure (described in the ``Calibration Cookbook'')
as follows:

\begin{verbatim}
        CLIC\SET SOURCE Phase_Calib RF_CAlib
        CLIC\SET PROC POINT CORR
        CLIC\SET FLUX Phase_Calib xx.x
        CLIC\SET FLUX RF_Calib yy.y
        CLIC\FIND 
        CLIC\STORE FLUX (On all cross-correlations) 
\end{verbatim}  
Now you can proceed to the standard Amplitude Calibration using
cross-correlations on the Amplitude Calibrator (see ``Calibration
Cookbook'').

\subsection{The Amplitude Calibrator is too weak for Pointings}

In this case, only the cross-correlations are used.  Don't forget to apply
the SOLVE FLUX command only on a short time interval where the weather
conditions: H$_2$O vapor contents and phase noise are quite similar.

\begin{verbatim}
   - Select the RF bandpass done usually after a FOCUS and a POINTING 
   - Select the adjacent Phase Calibrator(s)- it should follow a POINTING
\end{verbatim}

\begin{verbatim}
        CLIC\SET RF ON FREQ
        CLIC\SET PHASE RELATIVE
        CLIC\SET BAND DSB
        CLIC\SET SUB L01 TO L06
        CLIC\SET SOURCE Phase_Calib RF_Calib
        CLIC\SET FLUX RF_Calib xx.x     !see flux reports
        CLIC\SET PROC CORR
        CLIC\FIND /SCAN x y     !After a pointing on RF_Calib
        CLIC\FIND APPEND /SCAN v w      !After a pointing on Phase_Calib
        CLIC\SOLVE FLUX 
\end{verbatim}

When the flux density of the amplitude calibrator is known, update the {\tt
  CLIC} calibration procedure (described in the ``Calibration Cookbook'')
as follows:

\begin{verbatim}
        CLIC\SET SOURCE Phase_Calib RF_CAlib
        CLIC\SET PROC POINT CORR
        CLIC\SET FLUX Phase_Calib xx.x
        CLIC\SET FLUX RF_Calib yy.y
        CLIC\FIND 
        CLIC\STORE FLUX (On all cross-correlations) 
\end{verbatim} 
 
Now you can proceed to the standard Amplitude Calibration using
cross-correlations on the Amplitude Calibrator (see ``Calibration
Cookbook'').

\subsection{Using Pointings and Cross-correlations}

A new option in the command SOLVE FLUX has been implemented which allows
flux densities calculations using either POINTINGS and/or
CROSS-CORRELATIONS.  This is the command:
\begin{verbatim}
        CLIC\SOLVE FLUX /OUTPUT file NEW[OLD]
\end{verbatim}
This command, applied on the CROSS-CORRELATIONS, generates an output file
which can directly be read by he language FLUX in {\tt ASTRO}. It treats
the CROSS-CORRELATIONS like POINTINGS assuming that positionnal errors are
``zero''. The scenario is quite similar to the case above:

\begin{verbatim}
        CLIC\SET RF ON FREQ
        CLIC\SET PHASE RELATIVE
        CLIC\SET BAND DSB
        CLIC\SET SUB L01 TO L06
        CLIC\SET SOURCE Phase_Calib RF_CAlib
        CLIC\SET PROC CORR
        CLIC\SET FLUX RF_Calib xx.x     !see flux reports
        CLIC\FIND /SCAN x y     !After a pointing on RF_Calib
        CLIC\FIND APPEND /SCAN v w      !After a pointing on Phase_Calib
        CLIC\SOLVE FLUX/OUTPUT flux.fit NEW 
\end{verbatim}

When the flux density of the amplitude calibrator is known, update the {\tt
  CLIC} calibration procedure (described in the ``Calibration Cookbook'')
as follows:

\begin{verbatim}
        CLIC\SET SOURCE Phase_Calib RF_CAlib
        CLIC\SET PROC POINT CORR
        CLIC\SET FLUX Phase_Calib xx.x
        CLIC\SET FLUX RF_Calib yy.y
        CLIC\FIND 
        CLIC\STORE FLUX (On all cross-correlations) 
\end{verbatim} 
 
Now you can proceed to the standard Amplitude Calibration using
cross-correlations on the Amplitude Calibrator (see ``Calibration
Cookbook'').

\subsection{Use of W3OH as Flux Calibrator}

W3OH has been extensively and intensively studied with the interferometer.
Its flux is very well known between 84 and 115 GHz. Moreover its structure,
partially resolved by the interferometer (except on short baselines) is
easy to model. This is the reason why W3OH is observed about 10-15 minutes
in CROSS-CORRELATION mode in all porjects. The command SOLVE FLUX takes now
into account this model by comparing its flux to the predicted flux.  This
is very usefull to check that your flux determination is good.  Check it by
doing:
 
\begin{verbatim}
        CLIC\SET SOURCE W3OH
        CLIC\SET PROC CORR
        CLIC\FIND
        CLIC\SOLVE FLUX (On all cross-correlations) 
\end{verbatim} 


\section{The FLUX program}

FLUX is a superset of ASTRO which incorporates a new language (called FLUX)
to allow the derivation of fluxes from pointings scans made in single-dish
or interferometric modes. One has to use the following sequence:
 
\begin{verbatim}

FLUX\FILE filename
    - Define the entry file 'filename' (produced by PRINT FLUX/OUT)     

FLUX\READ antenna frequency 
    - Read the gaussian fits data from the antenna number 
      'antenna' and the frequency 'frequency'. 

ASTRO\SET FLUX name flux  
    - Attribute the flux value 'flux' to the source (usually a quasar)
      'name'. This optional command is used when no planet measurements is 
      available and/or as a redundant flux check on strong and well known 
      quasars.  Note that this is an ASTRO command.

FLUX\SOLVE name1 [name2 [...]]] 
    - Use the planet or quasar 'name1' as flux reference. For a planet,
      the flux is derived using the gain USB/LSB and the frequency 
      informations:
\end{verbatim}
\centerline{\bf $Flux(planet) = \frac{S(\nu_{lsb})+ g\times
    S(\nu_{usb})}{1+g}$ (Jy)}
\begin{verbatim}
      Then calculate the fluxes of all sources as described below at the 
      ``gain corrected'' frequency: 
\end{verbatim}
\centerline{\bf $\nu_{flux} = \frac{\nu_{lsb}+ g\times \nu_{usb}}{1+g}$
  GHz}
\begin{verbatim}
         + Correct Azimuth subscan for elevation pointing error
         + Correct Elevation subscan for azimuth pointing error
         + Average both subscans to derive the source intensity 
           (with proper weighting using standard deviations)
         + Average all independant scans on the same source (with 
           proper weighting)
         + Determine the  average antenna efficiency from the list 
           of sources of known flux, with error estimate
         + Determine the unknown fluxes, giving two error estimates:
           Relative (only the noise in the source measurement) and 
           Absolute (which also includes the error on antenna efficiency)

FLUX\PRINT /OUTPUT file [New]
    - Print the date and an information line on the meaning of each column,
      then the fluxes with error bars, the frequency, the antenna number
      and the average antenna efficiency (Jy/K), with error bars.
    - The file can be reopened.
 
FLUX\INDEX filei source/OUTPUT fileo [New]
    - Compute the spectral index of the source 'source' from all the
      data available on that source in the input file 'filei' (result from 
      FLUX\PRINT).
    - Write the results on the output file 'fileo'.
    - The output file can be reopened.

\end{verbatim}

\section{Possible Improvements}

A possible improvement would be consider systematically the error on the
flux determinations using the revised command: {\tt SET FLUX 'name' 'flux'
  'error'}, both in \astro\ and \clic.  The averaged antenna efficiencies
would be determined with errorbars, and fluxes of secondary calibrators
(SOLVE FLUX) would also include these errors.

\section{FLUX Language Internal Help}
\subsection{Language}
\index{Language}
\begin{verbatim}
              LLAS\ Command Language Summary
    (+ indicates a NEW command and - an obsolete command)

     LMETHOD    Selects operation do on spectra
     LRUN        do operation on spectra describe by LMETHOD
\end{verbatim}
\subsection{LMETHOD}
\index{LMETHOD}
\begin{verbatim}
         LLAS\LMETHOD [MERGE|...] FILE

    Selects operation do on spectra.
      MERGE     Merge hyperfine lines to improve S/N.
      FILE      File describes hyperfine structure. The first line is number
                of hyperfine lines (N). Each of the next N lines record
                velocity and relative intensity. Basicly same as the file
                format required by LAS\METHOD HFS.
\end{verbatim}
\subsection{LRUN}
\index{LRUN}
\begin{verbatim}
         LRUN

    Do operation on spectra describe by LMETHOD.
\end{verbatim}


\newpage \printindex{}

\end{document}

%%%%%%%%%%%%%%%%%%%%%%%%%%%%%%%%%%%%%%%%%%%%%%%%%%%%%%%%%%%%%%%%%%%%%%%%%%%
